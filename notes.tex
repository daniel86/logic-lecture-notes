\documentclass[10pt,a4paper]{article}
\usepackage[utf8]{inputenc}
\usepackage[T1]{fontenc}
\usepackage[ngerman]{babel}
\usepackage[sc,osf]{mathpazo}
\usepackage[german=guillemets]{csquotes}
\usepackage[fleqn]{amsmath}
\usepackage{amssymb,parskip,xspace,textcomp,latexsym,stmaryrd,enumitem}
\usepackage[margin={3cm,2cm}]{geometry}
\usepackage{tikz}
\usetikzlibrary{positioning}

\DeclareMathOperator{\AND}{\wedge}
\DeclareMathOperator{\OR}{\vee}
\DeclareMathOperator{\NOT}{\neg}
\DeclareMathOperator{\IMPL}{\rightarrow}
\DeclareMathOperator{\EQUIV}{\equiv}
\DeclareMathOperator*{\ANDop}{\bigwedge}
\DeclareMathOperator*{\ORop}{\bigvee}
\DeclareMathOperator{\UNION}{\cup}
\DeclareMathOperator{\SUBSET}{\subset}

\DeclareMathOperator{\Var}{\text{\textsf{Var}}}
\DeclareMathOperator{\Res}{\text{\textsf{Res}}}
\DeclareMathOperator{\ERes}{\text{\textsf{ERes}}}

\def\roem#1{\ensuremath{\text{#1}}\xspace}
\def\set#1{\ensuremath{\left\{#1\right\}}}
\def\I{\roem{I}}
\def\II{\roem{II}}
\def\III{\roem{III}}
\def\qed{\strut\\\null\hfill$\square$}
\def\QED{\strur\\\null\hfill$\blacksquare$}

\begin{document}

\section{VL vom 26.~Oktober 2010}

Wir verwenden die Variablen 

\begin{align}
  x_{ij}^\alpha \quad\text{mit}\quad \alpha \in \set{M,S,K}, i \in \set{\I,\II,\III}, j \in \set{a,b}
\end{align}

Zum Beispiel repräsentiert die Variable $x_{\II a}^M$ die Aussage
\enquote{Müller unterrichtet in Stunde \II Klasse $a$}.

Wir modellieren nun das Problem wie folgt:

\begin{itemize}
  \item Jede Stunde wurd von einem passendem Lehrer unterrichtet
  \begin{align}
    \varphi_1 &=
      \ANDop_{ij \in\set{\I a,\II a,\III b}} \biggl(x_{ij}^M \OR x_{ij}^K\biggr) \AND
      \ANDop_{ij \in\set{\I b,\II a,\II b}}  \biggl(x_{ij}^S \OR x_{ij}^K\biggr)
  \end{align}
  
  \item Jede Stunde wird von höchstens einem Lehrer unterrichtet:
  \begin{align}
    \varphi_2 &=
      \ANDop_{i\in\set{\I,\II,\III}} \ANDop_{j\in\set{a,b}}
        \NOT \biggl(x_{ij}^M \AND x_{ij}^S\biggr) \AND
        \NOT \biggl(x_{ij}^M \OR x_{ij}^K\biggr) \AND
        \NOT \biggl(x_{ij}^S \OR x_{ij}^K\biggr)
  \end{align}
  
  \item Jeder Lehrer unterrichtet mindesten zwei Stunden:
  \begin{align}
    \varphi_3 &=
      \ANDop_{\alpha\in\set{M,K,S}} \ORop_{\substack{i,i'\in\set{\I,\II,\III}\\ j,j'\in\set{a,b}\\ ij\not=i'j'}}
        \biggl(x_{ij}^\alpha \AND x_{i'j'}^\alpha\biggr)
  \end{align}
  
  \item Ein Lehrer kann nur eine Klasse zur Zeit unterrichten
  \begin{align}
    \varphi_4 &=
      \ANDop_{\alpha\in\set{M,K,S}} \ANDop_{i\in\set{\I,\II,\III}}
        \NOT \biggl(x_{ia}^\alpha \AND x_{ib}^\alpha\biggr)
  \end{align}
\end{itemize}

Man sieht nun leicht, dass die möglichen Lösungen für das Zeitplanungsproblem
ganau dem Belegungen $V$ entsprechen, die
$\varphi_1 \AND \varphi_2 \AND \varphi_3 \AND \varphi_4$ erfüllen.

\section{VL vom 28.~Oktober 2010}

\subsection{Ersetzungslemma}

\begin{description}
  \item[IA:]
  Wenn $\vartheta$ atomar ist, muss $\vartheta = \varphi$. Dann $\vartheta'=\psi$,
  also $\vartheta\EQUIV\vartheta'$ wegen $\psi\EQUIV\varphi$.

  \item[IS:]
  Wenn $\vartheta=\varphi$, argumentiere wir im IA. Sonst unterscheide drei Fälle:

  \begin{enumerate}
    \item $\vartheta = \NOT \vartheta_1$\\
    $\vartheta'$ hat die Form $\NOT\vartheta'_1$ (wobei sich $\vartheta'_1$ aus
    $\vartheta_1$ ergibt durch Ersetzen von $\varphi$ durch $\psi$. Nach IV gilt
    $\vartheta_1\EQUIV\vartheta'_1$, nach Semantik von \enquote{$\NOT$} also
    $\vartheta\EQUIV\vartheta'$.
    
    \item $\vartheta = \vartheta_1\OR\vartheta_2$\\
    $\varphi$ wird entweder in $\vartheta_1$ ider in $\vartheta_2$ durch $\psi$
    ersetzt. Wir betrachten nur den ersten Fall: dann
    $\vartheta'=\vartheta'_1\OR\vartheta_2$, also $\vartheta\EQUIV\vartheta'$
    
    \item $\vartheta = \vartheta_1\AND\vartheta_2$\\
    Analog zu 2.
  \end{enumerate}
  \qed
\end{description}


\subsection{Funktionale Vollständigkeit}

Die Funktionen in $\mathcal{B}^0$ werden durch die Formeln $0,1$ dargestellt.
Sei $n>0$ und $f\in\mathcal{B}^n$.

Für jedes $x\in Var$ sei $x^1=x$ und $x^0=\NOT x$. Für jedes Tupel
$f=(w_1,\dots,w_n)\in\set{0,1}^n$ sei $\varphi_f=x_1^{w_1}\AND\dots\AND x_n^{w_n}$.

\textbf{Definiere}
\begin{align}
  \varphi_f = \ORop_{\substack{f\in\set{0,1}^n\\ f(t) = 1}} \varphi_t
\end{align}

\textbf{Behauptung:} $f_{\varphi_f} = f$

Wenn $f(t) = 1$, dann ist $\varphi_t$ ein Disjunkt von $\varphi_f$. Also
$f_{\varphi_f}(t) = V_t(\varphi_f)=1$. Wenn umgekehrt $f_{\varphi_f}(t) = 1$,
dann $V_t(\varphi_f)=1$, also ist $\varphi_t$ Disjunkt von $\varphi_f$. Nach
Definition von $\varphi_f$ also $f(t)=1$. \qed


\subsection{Normalformen}

Sei $\varphi$ eine Formel. Äquivalente Formel in DNF ($\OR\AND$):

Konstruiere erst $f_\varphi$ mittels Wahrheitstafel, dann $\varphi_{f_\varphi}$
wie im vorigen Beweis. Das Resultat ist offensichtlich in DNF und die
Konstrktion ist effektiv.

Aus der effektiven Konstruierbarkeit der DNF folgt auch die der KNF ($\AND\OR$).

\begin{align}
  \varphi &\EQUIV \NOT\NOT \varphi                                 &&\text{DNF!}\\
          &\EQUIV \NOT\ORop_{i=1}^n \ANDop_{j=1}^{m_i} \ell_{i,j}  &&\text{de Morgan}\\
          &\EQUIV \ANDop_{i=1}^n \NOT\ANDop_{j=1}^{m_1} \ell_{i,j} &&\text{de Morgan}\\
          &\EQUIV \ANDop_{i=1}^n \ORop_{j=1}^{m_1} \NOT\ell_{i,j}  &&\text{KNF}
\end{align}
\qed

\section{VL vom 02. November 2010}

\subsection{Erfüllbarkeit ist in NP}

Sei $\phi$ Eingabeformel und sein $n = |\Var(\phi)|$.

Eine nicht-deterministische Touringmaschine kann mit $n$ nicht-determinischen
Übergängen eine Belegung für $\phi$ auf das Band schreiben. Danach prüft sie
deterministisch in Polynomialzeit, ob die geschriebene Belegung $\phi$ erfüllt.

Sie Aktzeptiert, wenn das der Fall ist und verwirft sonst. Offenbar löst die
Maschnine das Erfüllbarkeitsproblem in Polynomialzeit.

Erfüllbarkeit NP-hart $\leadsto$ VL Komplexitätstheorie (Cook's Theorem).

Gültigkeit: Reduktion auf Unerfüllbakeit 

\subsection{Korrektheit + Polynomialzeit für Horn-Formeln}

Offensichtlich terminiert der Algorithmus auf jeder Eingabe $\phi$ nach max.
$|\Var(\phi)|$ durchläufen der while-Schleife, also in Polynomialzeit.

Angenommen, der Algorithmus antowrtet \enquote{erfüllbar}. Dann gilt für die
konstruierte Menge $V$:

\begin{enumerate}
  \item Wenn $x_1\AND\dots\AND x_n \IMPL x$ Konjunkt von $\phi$ und $\set{x_1,\dots,x_n} \subseteq V$, dann $x \in V$.
  \item Wenn $x_1\AND\dots\AND x_n \IMPL 0$ Konjunkt von $\phi$, dann $\set{x_1,\dots,x_n} \not\subseteq V$.
\end{enumerate}

Also erfüllt $V$ (betrachtet als Belegung) $\phi$ und $\phi$ ist erfüllbar.

Angenommen, $\phi$ ist erfüllbar. Man zeigt leicht per Induktion über die Anzahl der Schleifendurchläufe:

\begin{itemize}
  \item[$*$] Wenn $x \in V$, dann $\hat{V}=1$ für alle Modell $\hat{V}$ von $\phi$.
\end{itemize}

Sei $x_1\AND\dots\AND x_n \IMPL 0$ Konjunkt von $\phi$. Es gilt
$\set{x_1,\dots,x_n} \not\subseteq V$, denn $\phi$ besitzt Modell $\hat{V}$ und
wenn $\set{x_1,\dots,x_n}\subseteq V$ ist mit $(*)$ auch
$\set{x_1,\dots,x_n} \subseteq \hat{V}$, im Widerspruch dazu dass $\hat{V}$ Modell
von $\phi$ ist. \qed

\subsubsection{Beispiel}

\[
  V=\set{\text{Regen}, \text{Schnee}} \cup \set{\text{Niederschlag}, \underline{\text{Temp}>0}, \underline{\text{Temp}\leq 0}}
\]

\section{VL vom 09. November 2010}

\subsection{Beweis Resolutionslemma}

\begin{itemize}
  \item $V \models  M \cup \set{C} \Rightarrow V \models  M$ trivial.
  \item Es gelte $V\models M$. Ferner sei $C=(C_1\backslash \set{\ell}) \cup
  (C_2\backslash \set{\overline{\ell}})$ mit $C_1,C_2 \in M$. Unterscheide zwei
  Fälle:
  \begin{itemize}
    \item $V(\ell) = 1$. Wegen $V\models C_2$ dann auch $V\models C_2\backslash \set{\overline{\ell}}$.
    \item $V(\ell) = 0$. Wegen $V\models C_1$ dann auch $V\models C_1\backslash \set{\overline{\ell}}$.
  \end{itemize}
  In beiden Fällen also $V\models C$.\qed
\end{itemize}

\subsubsection{Beispiel-Resolution}

\begin{align}
  M(\varphi) &= \set{\set{x_1}, \set{\NOT x_1, x_2}, \set{\NOT x_2, x_3}, \set{\NOT x_3}} = M \\
  \Res^0(M)   &= M \\
  \Res^1(M)   &= \Res^0(M) \cup \set{\set{x_2}, \set{\NOT x_1, x_3}, \set{\NOT x_2}} \\
  \Res^2(M)   &= \Res^1(M) \cup \set{\set{x_3}, \set{\NOT x_1}, \square} \\
  \Res^3(M)   &= \Res^2(M) = \Res^*(M)
\end{align}

\subsection{Beweis Resolutionssatz}

\begin{itemize}
  \item[$\Leftarrow$] \textbf{Korrektheit}\par
  Da $\square \in \Res^*(M)$, ist $\Res^*(M)$ unerfüllbar. Es genügt also zu zeigen, dass $\Res^*(M)\EQUIV M$.
  Mittels Resolutionslemma und per Induktion über $i$ zeigt man leicht, dass $M\EQUIV \Res^i(M) \forall i \geq 0$.
  Da sich über den endlich vielen Literalen in $M$ nur endlich viele Klauseln bilden lassen, ist $\Res^*(M)$ endlich, also $\Res^*(M) = \Res^i(M)$ für ein $i\geq 0$, damit $M\EQUIV \Res^*(M)$.
  
  \item[$\Rightarrow$] \textbf{Vollständigkeit}\par
  Wir zeigen
  \[
    M\ \text{unerfüllbar} \IMPL \square \in \Res^*(M)
  \]
  per Induktion über $|\Var(M)|$.
  \begin{description}
    \item[IA] $\Var(M) \not= \emptyset$. Dann $M=\emptyset$ oder $M=\set{\square}$.
    Da $\emptyset$ erfüllbar, muss $M=\set{\square}$ sein, also $\square\in M \subseteq \Res^*(M)$.
    
    \item[IS] Wähle $x\in \Var(M)$, konstruiere zwei Klauselmengen:
    \begin{align}
      K^+ &:= \set{C \backslash \set{\NOT x} | x \not\in C \in M} \\
      K^- &:= \set{C \backslash \set{x} | \NOT x \not\in C \in M} 
    \end{align}
    Intuitiv entspricht $K^+$ dem Fall $V(x) = 1$: alle Klauseln $C$ mit $x \in C$
    sind erfüllt und wurden gestrichen, aus den verbliebenen Klauseln kann
    $\NOT x$ gestrichen werden (wenn vorhanden), denn $\NOT x$ kann die Klausel
    nicht wahr machen.
    
    Wir zeigen:
    \begin{enumerate}
      \item $K^+$ und $K^-$ sind unerfüllbar
      \item $\square \in \Res^*(M)$ oder $\set{\NOT x} \in \Res^*(M)$
      \item $\square \in \Res^*(M)$ oder $\set{x} \in \Res^*(M)$
    \end{enumerate}
    
    \item[(1)] Angenommen, $K^+$ ist erfüllbar und $V\models K^+$.
    Erweitere $V$ durch $V(x)=1$. Man prüft leicht, dass $V\models M$. $\lightning$
    
    Unerfüllbarkeit $K^-$ analog.
    
    \item[(2)] Weil $K^+$ unerfüllbar liefert IV $\square\in\Res^*(K^+)$.
    Also gibt es Klauseln $C_1,\dots,C_m$, so dass $C_m=\square$ und für $1\leq i\leq m$ gilt
    \begin{enumerate}
      \item $C_i \in K^+$ oder
      \item $C_i$ ist Resolvente von $C_j,C_k$ für je gewisse $j,k < i$.
    \end{enumerate}

    \textbf{Fall 1:} alle Klauseln $C_i$ der Form (a) sind auch in $M$ (in keiner der
    \enquote{Originalklauseln} kam $\NOT x$ vor). Dann prüft man leicht, dass
    $C_1,\dots,C_m \in\Res^*(M)$, also $\square\in\Res^*(M)$.
    
    \textbf{Fall 2:} Für mind. ein $C_i$ der Form (a) ist $C_i\cup\set{\NOT x}\in M$.
    Wir erhalten duch Wiedereinfügen von $\NOT x$ eine Folge von Klauseln
    $C'_1,\dots,C'_m \in \Res^*(M)$, die beweist, dass $\set{\NOT x}\in\Res^*(M)$.
    
    \begin{center}
      \begin{tikzpicture}
        \node (j) at (-1,1) {$C_j$};
        \node (k) at (1,1) {$C_k$};
        \node (i) at (0,0) {$C_i$};
        \draw (j)--(i) (k)--(i);
        
        \node at (3,0.5) {$\Rightarrow$};
        \node (j) at (5,1) {$C_j\cup\set{\NOT x}$};
        \node (k) at (9,1) {$C_k\cup\set{\NOT x}$};
        \node (i) at (7,0) {$C_i\cup\set{\NOT x}$};
        \draw (j)--(i) (k)--(i);
      \end{tikzpicture}
    \end{center}
    \item[(3)] Analog zu (2) unter Verwendung von $K^-$.
  \end{description}

  Aus (2) und (3) folgt, dass $\square\in\Res^*(M)$ oder $\set{x},\set{\NOT x}\in\Res^*(M)$. Mit
  \begin{center}
    \begin{tikzpicture}
      \node (j) at (-1,1) {$\set{\NOT x}$};
      \node (k) at (1,1) {$\set{x}$};
      \node (i) at (0,0) {$\square$};
      \draw (j)--(i) (k)--(i);
    \end{tikzpicture}
  \end{center}
  dass auch $\square\in\Res^*(M)$.\qed
\end{itemize}

\subsection{Beispiel: Einheitsresolvente}

\begin{center}
  \begin{tikzpicture}
    \node             (a) {$\set{\NOT x_1, \NOT x_2, \NOT x_3, x_4}$};
    \node[right=of a] (b) {$\set{x_1}$};
    \node[right=of b] (c) {$\set{x_2}$};
    \node[right=of c] (d) {$\set{x_3}$};
    \node[right=of d] (e) {$\set{\NOT x_3, \NOT x_4}$};
    \node[xshift=4em,below=of a] (f) {$\set{\NOT x_2, \NOT x_3, x_4}$};
    \node[xshift=5em,below=of f] (g) {$\set{\NOT x_3, x_4}$};
    \node[xshift=5em,below=of g] (h) {$\set{x_4}$};
    \node[xshift=5em,below=of h] (i) {$\set{\NOT x_3}$};
    \node[xshift=6em,below=of i] (j) {$\square$};
    
    \draw (a)--(f) (b)--(f)
      (c)--(g) (f)--(g)
      (d)--(h) (g)--(h)
      (e)--(i) (h)--(i)
      (d)--(j) (i)--(j);
  \end{tikzpicture}
\end{center}

\section{VL vom 11. November 2010}




\end{document}


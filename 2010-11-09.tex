\section{VL vom 09. November 2010}

\subsection{Beweis Resolutionslemma}

\begin{itemize}
  \item $V \models  M \cup \set{C} \Rightarrow V \models  M$ trivial.
  \item Es gelte $V\models M$. Ferner sei $C=(C_1\backslash \set{\ell}) \cup
  (C_2\backslash \set{\overline{\ell}})$ mit $C_1,C_2 \in M$. Unterscheide zwei
  Fälle:
  \begin{itemize}
    \item $V(\ell) = 1$. Wegen $V\models C_2$ dann auch $V\models C_2\backslash \set{\overline{\ell}}$.
    \item $V(\ell) = 0$. Wegen $V\models C_1$ dann auch $V\models C_1\backslash \set{\overline{\ell}}$.
  \end{itemize}
  In beiden Fällen also $V\models C$.\qed
\end{itemize}

\subsubsection{Beispiel-Resolution}

\begin{align}
  M(\varphi) &= \set{\set{x_1}, \set{\NOT x_1, x_2}, \set{\NOT x_2, x_3}, \set{\NOT x_3}} = M \\
  \Res^0(M)   &= M \\
  \Res^1(M)   &= \Res^0(M) \cup \set{\set{x_2}, \set{\NOT x_1, x_3}, \set{\NOT x_2}} \\
  \Res^2(M)   &= \Res^1(M) \cup \set{\set{x_3}, \set{\NOT x_1}, \square} \\
  \Res^3(M)   &= \Res^2(M) = \Res^*(M)
\end{align}

\subsection{Beweis Resolutionssatz}

\begin{itemize}
  \item[$\Leftarrow$] \textbf{Korrektheit}\par
  Da $\square \in \Res^*(M)$, ist $\Res^*(M)$ unerfüllbar. Es genügt also zu zeigen, dass $\Res^*(M)\EQUIV M$.
  Mittels Resolutionslemma und per Induktion über $i$ zeigt man leicht, dass $M\EQUIV \Res^i(M) \forall i \geq 0$.
  Da sich über den endlich vielen Literalen in $M$ nur endlich viele Klauseln bilden lassen, ist $\Res^*(M)$ endlich, also $\Res^*(M) = \Res^i(M)$ für ein $i\geq 0$, damit $M\EQUIV \Res^*(M)$.
  
  \item[$\Rightarrow$] \textbf{Vollständigkeit}\par
  Wir zeigen
  \[
    M \text{unerfüllbar} \IMPL \square \in \Res^*(M)
  \]
  per Induktion über $|\Var(M)|$.
\end{itemize}

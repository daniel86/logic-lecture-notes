\section{VL von 08.~Februar 2011}

\subsection{Hamiltonkreis}

Angenommen, $\Afrak$ enthält Hamiltonkreis $a_1,\dots,a_n$.

Setze 
\begin{align*}
  S^\Afrak &= \set{(a_i,a_{i+1} \mid 1\leq i\leq n,\ \text{wobei}\ a_{n+1} := a_1} \\
  L^\Afrak &= \text{transitive Hülle von}\ S\backslash\set{(a_n,a_1)}
\end{align*}

Man überprüft leicht, dass alle FO-Formeln in $\varphi_\HK$ erfüllt sind,
also $\Afrak\models\varphi_\HK$.

Angenommen, $\Afrak\models\varphi_\HK$. Seien $L^\Afrak$ und $S^\Afrak$
Relationen, so dass alle FO-Formeln in $\varphi_HK$ erfüllt sind.
Damit ist $S^\Afrak$ ein Hamiltonkreis:

\begin{itemize}
  \item $S^\Afrak$ ist ein Kreis (3. Konjunkt)
  \item $S^\Afrak$ enthält nur Kanten aus $E$ (4. Konjunkt)
  \item $S^\Afrak$ enthält alle Elemente, weil $L$ alle Elemente enthält (2. Konjunkt)
  \item jedes Element taucht höchsten einmal in $S^\Afrak$ auf, da $S^\Afrak$
  Nachfolgerelation einer (zyklenfreien!) linearen Ordnung ist.
\end{itemize}
\qed

\section{VL vom 02. November 2010}

\subsection{Erfüllbarkeit ist in NP}

Sei $\phi$ Eingabeformel und sein $n = |\Var(\phi)|$.

Eine nicht-deterministische Touringmaschine kann mit $n$ nicht-determinischen
Übergängen eine Belegung für $\phi$ auf das Band schreiben. Danach prüft sie
deterministisch in Polynomialzeit, ob die geschriebene Belegung $\phi$ erfüllt.

Sie Aktzeptiert, wenn das der Fall ist und verwirft sonst. Offenbar löst die
Maschnine das Erfüllbarkeitsproblem in Polynomialzeit.

Erfüllbarkeit NP-hart $\leadsto$ VL Komplexitätstheorie (Cook's Theorem).

Gültigkeit: Reduktion auf Unerfüllbakeit 

\subsection{Korrektheit + Polynomialzeit für Horn-Formeln}

Offensichtlich terminiert der Algorithmus auf jeder Eingabe $\phi$ nach max.
$|\Var(\phi)|$ durchläufen der while-Schleife, also in Polynomialzeit.

Angenommen, der Algorithmus antowrtet \enquote{erfüllbar}. Dann gilt für die
konstruierte Menge $V$:

\begin{enumerate}
  \item Wenn $x_1\AND\dots\AND x_n \IMPL x$ Konjunkt von $\phi$ und $\set{x_1,\dots,x_n} \subseteq V$, dann $x \in V$.
  \item Wenn $x_1\AND\dots\AND x_n \IMPL 0$ Konjunkt von $\phi$, dann $\set{x_1,\dots,x_n} \not\subseteq V$.
\end{enumerate}

Also erfüllt $V$ (betrachtet als Belegung) $\phi$ und $\phi$ ist erfüllbar.

Angenommen, $\phi$ ist erfüllbar. Man zeigt leicht per Induktion über die Anzahl der Schleifendurchläufe:

\begin{itemize}
  \item[$*$] Wenn $x \in V$, dann $\hat{V}=1$ für alle Modell $\hat{V}$ von $\phi$.
\end{itemize}

Sei $x_1\AND\dots\AND x_n \IMPL 0$ Konjunkt von $\phi$. Es gilt
$\set{x_1,\dots,x_n} \not\subseteq V$, denn $\phi$ besitzt Modell $\hat{V}$ und
wenn $\set{x_1,\dots,x_n}\subseteq V$ ist mit $(*)$ auch
$\set{x_1,\dots,x_n} \subseteq \hat{V}$, im Widerspruch dazu dass $\hat{V}$ Modell
von $\phi$ ist. \qed

\subsubsection{Beispiel}

\[
  V=\set{\text{Regen}, \text{Schnee}} \cup \set{\text{Niederschlag}, \underline{\text{Temp}>0}, \underline{\text{Temp}\leq 0}}
\]
